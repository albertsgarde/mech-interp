\Acp{LLM} based on the transformer architecture \parencite{vaswani_attention_2023} have shown exceptional performance across a range of tasks, yet how they achieve such impressive results remains poorly understood.
As these models are increasingly deployed in real-world applications, concerns have been raised about the potential for harm \parencite{noauthor_statement_nodate}\parencite{hendrycks_overview_2023}.
These include the potential for bias, the potential for misuse by malicious actors, and existential risks from misaligned models \parencite{ngo_alignment_2024}.
This makes the opacity of these models a significant issue, since only having a black-box understanding of their behaviour limits our ability to foresee or guarantee against harmful behaviour in out of distribution scenarios.
Additionally, a greater understanding of these models may allow us to better correct these behaviours once they are identified.
The fields of interpretability and explainability aim to address this issue by providing insights into the internal workings of these models.
Many approaches have been developed within these fields.
\textcite{bereska_mechanistic_2024} provides a recent overview and a useful taxonomy, while \textcite{rauker_toward_2023} provides a more thorough survey including a list of previous reviews in the same area.
%While most work is focused on empirical work, \textcite{elhage_mathematical_2021} gives a theoretical perspective on the topic.

Transformer models contain both attention and \ac{MLP} layers and the latter is the focus of this thesis.
Previous attempts at interpreting \ac{MLP} neurons have focused on understanding the behaviour of individual neurons \parencite{wang_interpretability_2022}, but as demonstrated in \textcite{elhage_toy_2022}, the behaviour of individual neurons often doesn't map onto human understandable concepts.
\textcite{bricken_towards_2023} shows a possible way forward by suggesting the latents of \acp{SAE} trained on the \ac{MLP} layer activations as alternative units of interpretability.
In this work, we focus on these along with the \ac{N2G} \parencite{foote_neuron_2023} method.
This is an interesting combination, since if \acp{SAE} truly do provide more interpretable features and \ac{N2G} truly does provide a useful representation of feature behaviour, we would expect this to be reflected when comparing \ac{N2G} graphs for individual neurons against those for \ac{SAE} latents.
Indeed, \textcite{gao_scaling_2024} uses \ac{N2G} to test the interpretability of the latents of their \acp{SAE}.

In this work we first provide a review of the literature on \acp{SAE} as applied to interpretability of transformer models as well as giving a description of the \ac{N2G} method.
The review differs from the reviews mentioned above by focussing exclusively on \acp{SAE} and including the most recent work.
To our knowledge there is no other review that does this.
This allows us to go into considerable detail, and since the field is so young, it means we can cover essentially all relevant work.
We then move on to our own experiments, where our main goal is to explore the clusters found in the \ac{SAE} latents by \textcite{bricken_towards_2023}.
To do so we measure the density of both \ac{SAE} latents and \ac{MLP} neurons and estimate their interpretability using the proxy of \ac{N2G} performance.
However, to start, we replicate the main finding of \textcite{bricken_towards_2023} and \textcite{cunningham_sparse_2023} that the \ac{SAE} latents are more interpretable than the \ac{MLP} neurons.
Finally, we investigate the hypothesis that language models exploit the exponentially many nearly orthogonal directions in the \ac{MLP} activation space to encode information.
We do this by looking at the encoding vectors for the \ac{SAE} latents.

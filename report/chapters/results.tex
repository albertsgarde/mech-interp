In this chapter we present the results of our experiments, starting with an investigation of the performance of \ac{N2G} on the two populations of features: \ac{MLP} neurons and \ac{SAE} latents.
We then look at the relationship between feature \ac{N2G} performance and density.
This gives rise to an intuitive clustering of features based on these two statistics, which are defined as the high density, low density cluster, and interpretable clusters.
This clustering is also related to both the geometry of the features and the sizes of the \ac{N2G} graphs, which we investigate in the last two sections of this chapter.
In all cases, the focus is on the direct results of our experiments and not any broader conclusions.
This we leave for the discussion in \autoref{sec:discussion}

As described in section \ref{sec:preliminaries}, features are functions of the \ac{LM}'s activations.
In the following, two sets of features are studied: \ac{MLP} neurons and \ac{SAE} latents.
"Features" here refers to the union of the two sets while "neurons" and "latents" refer to the individual sets.



\begin{table}[ht]
    \centering
    \input{images/figures/distribution_table.tex}
    \caption{Means and standard deviations for the statistics (N2G performance and feature density) of the two populations. Only includes features with a non-nan F1-score and a nonzero density. According to a two-sample bootstrap test, the distribution means for all statistics are different with $p<0.0001$.}
    \label{tab:distributions}
\end{table}

\begin{figure}[ht]
    \centering
    
    \begin{subfigure}[b]{0.45\textwidth}
        \centering
        \includegraphics[width=\textwidth]{images/figures/distribution_recall.pdf}
        \caption{Recall Distribution}
        \label{fig:distributions_recall}
    \end{subfigure}
    \begin{subfigure}[b]{0.45\textwidth}
        \centering
        \includegraphics[width=\textwidth]{images/figures/distribution_precision.pdf}
        \caption{Precision Distribution}
        \label{fig:distributions_precision}
    \end{subfigure}
    
    \begin{subfigure}[b]{0.45\textwidth}
        \centering
        \includegraphics[width=\textwidth]{images/figures/distribution_f1.pdf}
        \caption{F1 Distribution}
        \label{fig:distributions_f1}
    \end{subfigure}
    \begin{subfigure}[b]{0.45\textwidth}
        \centering
        \includegraphics[width=\textwidth]{images/figures/distribution_log_density.pdf}
        \caption{Density Distribution}
        \label{fig:distributions_log_density}
    \end{subfigure}
    
    \caption{Distributions of our statistics across the two populations. 
    Density is calculated with a activation threshold of $0.6$. Since for some of the features the \ac{N2G} model never predicts an activation, the precision and F1 is sometimes undefined. A value of NaN is used to indicate this.}
    \label{fig:distributions}
\end{figure}

\section{N2G Performance}
The analysis begins with an examination of the performance of the \ac{N2G} models across the two populations: \ac{MLP} neurons and \ac{SAE} latents.
For each feature, we calculate the recall, precision, and F1-score of the \ac{N2G} model trained on that feature along with the density of that feature, i.e. on what proportion of tokens does its activation exceed some threshold.
\autoref{tab:distributions} and \autoref{fig:distributions} show the distributions of these statistics across the two populations, with the activation threshold set to $0.6$.
A two-sample bootstrap test shows that the means of the distributions are different for all statistics with $p<0.0001$.
More interesting is what the plots show about the nature of this distribution difference.
Looking at the plots of the performance metrics, it seems that the two populations share a mode at roughly $0.1$, but the \ac{SAE} population has a second mode at almost $1$.
This indicates that while most \ac{SAE} latents are as badly modelled by \ac{N2G} as the \ac{MLP} neurons, there is a cluster of \ac{SAE} latents that are modelled almost perfectly.
\autoref{fig:recall_precision} confirms this by showing that there is a set of \ac{SAE} latents with both near perfect recall and near perfect precision.

\begin{figure}[ht]
    \centering
    
    \begin{subfigure}[b]{0.45\textwidth}
        \centering
        \includegraphics[width=\textwidth]{images/figures/recall_precision_mlp.pdf}
        \caption{Recall vs Precision for \ac{MLP}}
        \label{fig:recall_precision_mlp}
    \end{subfigure}
    \begin{subfigure}[b]{0.45\textwidth}
        \centering
        \includegraphics[width=\textwidth]{images/figures/recall_precision_sae.pdf}
        \caption{Recall vs Precision for SAE}
        \label{fig:recall_precision_sae}
    \end{subfigure}
    
    \caption{Distributions of our statistics across the two populations. Only includes features with a non-nan F1-score and a nonzero density.}
    \label{fig:recall_precision}
\end{figure}

\section{Density and performance}
Density also has a bimodal distribution for \acp{SAE}, though here neither mode lines up well with the \ac{MLP} distribution.
This begs the question of whether the modes of the performance metric distributions are related to the modes of the density distribution.
\autoref{fig:density_f1_cluster_check} sheds light on this question.
\autoref{fig:density_f1} shows that while some features with poorly performing \ac{N2G} models ($F1<0.5$) are in the low density cluster, none of the well-performing features are in the high density cluster.
Likewise, \autoref{fig:f1_density} shows that while low density features are spread over much of the \ac{N2G} performance spectrum, the dense features are almost entirely located in the low performing cluster.
Altogether this means that low density is a necessary but not sufficient condition for a feature to be well modelled by \ac{N2G}.
In other words, the high F1 cluster is a subset of the low density cluster.
This is backed up by the \texttt{N} column of \autoref{tab:directions} which confirms that almost all high F1 features are low density.

\begin{figure}[ht]
    \centering
    
    \begin{subfigure}[b]{0.45\textwidth}
        \centering
        \includegraphics[width=\textwidth]{images/figures/density_f1.pdf}
        \caption{Density in each F1 cluster}
        \label{fig:density_f1}
    \end{subfigure}
    \begin{subfigure}[b]{0.45\textwidth}
        \centering
        \includegraphics[width=\textwidth]{images/figures/f1_density.pdf}
        \caption{F1 in each density cluster}
        \label{fig:f1_density}
    \end{subfigure}
    
    \caption{The relationship between density and F1-score for \ac{SAE} features.
    Each plot splits the features into clusters based on one metric and shows the distribution of the other within each cluster.
    The cluster boundaries are chosen based on figure \ref{fig:distributions_f1} and \ref{fig:distributions_log_density}.}
    \label{fig:density_f1_cluster_check}
\end{figure}

\section{Geometry}
We have shown a strong relationship between density and \ac{N2G} performance.
Both of these are empirical measures of the behaviour of the feature on a dataset.
This raises a question of whether these clusters are also meaningful directly in terms of the feature directions (see \autoref{sec:methods_geometry}) read directly from the $\mat W_e$ encoding matrix in the SAE.
It turns out the answer is yes.
Specifically, it turns out that features in the high density cluster all represent directions that are very similar, i.e. they all point in the same direction.
This can be seen in \autoref{tab:directions} where the high density cluster has a mean cosine similarity of $0.96$, and a variance of $0.04$.
Meanwhile, the other clusters have a mean cosine similarity less than $0.02$ and a variance of at least $0.98$.
The variance of the raw high density feature directions implies that though the vectors point in roughly the same direction, they still vary significantly in magnitude.

Together with the results of the previous paragraph, this suggests that there are three meaningful clusters of features: a high density, low F1 cluster, a low density, high F1 cluster, and "the rest", a low density, low F1 cluster.
The high density cluster is characterized by all features pointing in the same direction, while the low density, high F1 cluster is characterized by a spread of directions.

\begin{table}[ht]
    \centering
    \input{images/figures/direction_table.tex}
    \caption{Results of statistical analysis on the directions of various feature clusters.
    The clusters are the same as those used in previous figures (log density threshold is $-1.2$ and F1 threshold is $0.5$), except the random cluster which is a set of directions randomly generated according to the standard $2048$-dimensional normal distribution.
    \texttt N is the number of features in the cluster.
    \texttt{Cos sim.} is the mean pairwise cosine similarity of directions in the cluster.
    \texttt{Variance} is the trace of the covariance matrix and is split into a raw and a normalized column, where the former is the value for the directions taken directly from the \ac{SAE} while for the latter the directions have been normalized to norm $1$.
    }
    \label{tab:directions}
\end{table}

\section{N2G size}
The clusters found above also have a clear relationship to the size of the \ac{N2G} graphs.
\autoref{tab:n2g_sizes} shows the quartiles of the \ac{N2G} graph sizes for the different feature clusters.
It shows a clear difference in graph sizes, with an ordering in all quartiles of high density $<$ MLP $<$ low density $<$ high F1.
This is confirmed by a two sample bootstrap test, where within each quartile, all pairs are significantly different with $p<0.0001$.
The only exception is the first quartile where both the low density cluster and the high F1 cluster have a Q1 of $1$ in all bootstrap samples.

\begin{table}[ht]
    \centering
    \input{images/figures/n2g_size_table.tex}
    \caption{Quartiles of \ac{N2G} graph sizes for the different feature clusters. The graph size is measured as the number of nodes in the trie, not the number of nodes in the vizualized graph.
    All differences are significant in a two sample bootstrap test with $p<0.0001$.
    }
    \label{tab:n2g_sizes}
\end{table}
In this chapter we describe the methods used in this work.
We first describe what SAEs we use.
Then we describe which text dataset we use, and how we obtain the samples needed for N2G from that dataset.
Afterwards we go into detail about the implementation of the N2G method, including the code changes needed from the original implementation as well as how we evaluate the resulting N2G models.
Finally we describe the statistical methods used to analyse the data gained from the features and from evaluating the N2G models.
Most code used in this work is available at \footnote{\url{https://github.com/albertsgarde/thesis}} while our modifications to the N2G code can be found at \footnote{\url{https://github.com/albertsgarde/Neuron2Graph}}.

\section{SAEs}
Since SAEs are finnicky to train, we will be using a pre-trained SAE, namely that provided by \textcite{nanda_neelnandasparse_autoencoder_nodate}.
Though the past months have seen the development of several open source SAE libraries \parencite{bloom_jbloomaussaelens_2024}\parencite{cooney_ai-safety-foundationsparse_autoencoder_2024}, we started this work before then and do not use any of them.
Attempts were made to extend the experiments to SAEs on \texttt{gpt2-small}, specifically those provided by \textcite{wu_openaisparse_autoencoder_2024}, but they proved unsuccesful due to memory constraints making the process too time consuming.

\section{Dataset}
\label{sec:dataset}
Our experiments require data in the form of a number of samples for each feature.
The text dataset we use is the same as was originally used to train both the LM and the SAE, namely \textcite{nanda_neelnandac4-code-20k_nodate}.
The dataset contains samples of varying lengths, but we need fixed lengths samples for both more convenient and more efficient processing.
We solve this by creating overlapping samples of fixed length within each dataset sample, padding with the padding token when necessary.
The overlap is to ensure that all tokens are included with some preceding context.

\subsection{Sampling}
\label{sec:sampling}
For N2G to run on a feature, it needs a set of samples that activate the feature.
Since the running time scales linearly in the number of samples, we would like to keep this number as low as possible.
At the same time, what samples N2G is run on has a large impact on the results.
In the original N2G paper \parencite{foote_neuron_2023}, maximum activating samples were used.
However, we found that this leads to many of the samples being very similar which pollutes the test set.
Because of this and arguments that maximum activating samples can be misleading \parencite{bolukbasi_interpretability_2021}, we instead use a weighted random sampling approach.
Like with maximum activating samples we assume that the samples that activate the feature most highly are the most important for understanding the feature, but we do not assume that they are the \emph{only} important samples.
Inspired by softmax, we assign each sample a weight equal to
\begin{align*}
    w=\e^{\alpha a}
\end{align*}
where $a$ is the maximum activation of the feature on that sample and $\alpha$ is a hyperparameter controlling how much to prioritize the most activating samples.
A \emph{key} is then calculated for each sample given by
\begin{align*}
    k=\xi^{\frac1w}-[a<c]
\end{align*}
where $\xi\sim\mathrm{Uniform}(0,1)$ is a random number and $[a<c]$ is $1$ when the activation is below a firing threshold and $0$ otherwise.
Since the first term is always between $0$ and $1$, the second term ensures that samples with activations above the threshold are always prioritized.
Sampling is then done by streaming through a part of the dataset, calculating the key for each sample, and keeping the $n$ samples with the highest keys.
The second term ($[a<c]$) was included after noticing that some samples with all zero activations were included, which is a waste since they do not contribute to either training or evaluation.
The implementation used for these experiments is available \footnote{\url{https://github.com/albertsgarde/thesis/blob/main/thesis/mas/weighted_samples_store.py}} in the repo.
For the results below, we used $\alpha=1$, $c=0.5$, and $n=32$.

\section{N2G}
Our code for training and evaluating N2G models is heavily based on the original implementation \parencite{foote_apartresearchneuron2graph_2024}.
However the original was a single script with no documentation and since we needed to work in depth with the code we have done considerable refactoring work to improve code quality.
In our fork, the code has a more modular structure, all functions are typed, all constants are configurable from outside the code, and we follow the standards enforced by Ruff \parencite{marsh_astral-shruff_2024}.
During this work we found many redundant calculations, the removal of which has likely sped up the code though we have not done any formal benchmarking.
We also found that the original representation of the N2Gs took up so much memory that our computers often struggled to load all at once and they took up a lot of space on disk.
While the disk usage could be solved by compressing the files (N2G graphs compress very well), this would not solve the memory usage.
Our solution was to create a Rust implementation of the N2Gs.
While the memory impact is non-trivial to measure, the compressed disk size is not and it fell by a factor of $10$.
This implementation does not support training, but can be used to evaluate N2Gs and produce the visual graph representations.
The algorithm thus works by first training the N2Gs using the Python implementation, before converting them to the Rust implmentation for storage.

\subsection{Evaluation}
Given a feature and a N2G model trained on that feature's training set, we evaluate the model on the test set.
This is done by getting it to predict for each token in each sample of the test set whether the feature will fire on that token, where firing is defined as the activation being above a threshold.
These predictions can then be compared to the actual activations of the feature to calculate recall, precision, and F1-score.

\section{Analysis}
\todo{Explain cosine similarity and variance with formulas}
After N2G models have been trained and evaluated for all MLP and SAE features, we can compare the two populations.
Our analysis is based on 4 main statistics for each feature: recall, precision, F1-score, and feature density.
The first three refer to the performance of the N2G model of the feature when predicting the feature's activations on the test set while the last one is the fraction of tokens that activate the feature.
To compare these statistics between the two populations, we implement two-sample bootstrap tests.
After defining a few ad-hoc clusters based on these statistics, we investigate the geometric properties of the features in these clusters.
Specifically, we look at the encoding directions of the SAE latents and compare statistical properties of directions in the clusters.
All these concepts are described in more detail in the following subsections.

\subsection{Density}
\label{sec:density}
To calculate density, we first define a threshold for what counts as an "activating" feature.
In this work, we choose a threshold of $0.6$.
This is somewhat arbitrary, but seems like a good balance between including high activations and excluding noise.
Experiments with other thresholds such as $0.2$ showed that the results were not particularly sensitive to the choice of threshold.
After defining a threshold, we run the LM on the dataset and count the number of tokens where each feature has an activation above the threshold.
The density of a feature is then the number of such tokens divided by the total number of tokens in the dataset.

\subsection{Geometry}
\label{sec:geometry}
The "direction" of an SAE latent is the vector in MLP activation space that the latent represents.
If we take the mathematical representation of an SAE \eqref{eq:sae_structure} 
\begin{align*}
    \vec y=&\mathrm{ReLU}\left(\mat W_e\vec x+\vec b\right)\\
    \widehat{\vec x}=&\mat W_d\vec y
\end{align*}
the direction of the $i$'th latent is $\mat W_e[i]$, i.e. the $i$'th row of the encoding matrix $\mat W_e$.
There are arguments to be made\parencite{nanda_open_2023} that the respective row in the decoder matrix $\mat W_d$ is more representative of the latent, but we have chosen to use the encoding matrix for this work in keeping with our theme of looking at the behaviour of the features rather than their effect on the LM.

Since these are directions in a high-dimensional space, we cannot simply visualize them or reason intuitively about them.
Instead we use two statistics to boil down interesting properties to a few numbers we can understand: mean cosine similarity and variance.

\subsubsection{Cosine similarity}
Cosine similarity is a measure of how similar the directions of two vectors are and is defined for two non-zero vectors $\vec a,\vec b\in\R^{N}\setminus \{\vec 0\}$ in an $N\in\N$ dimensional space as
\begin{align*}
    \cos(\vec a,\vec b)=\frac{\vec a\cdot\vec b}{\norm{\vec a}\norm{\vec b}}=\widehat{\vec a}\cdot\widehat{\vec b}
\end{align*}
where $\widehat{\vec x}$ denotes the normalized vector $\vec x$.
Since matrix multiplication is defined as the dot product of the rows of the first matrix with the columns of the second, finding the cosine similarity between all of latent directions can be written as
\begin{align*}
    \widehat{\mat W_e}\trans{\widehat{\mat W_e}}
\end{align*}
where $\widehat{\mat A}$ is the matrix where each row is normalized.
The mean of these values gives a measure of how similar the directions are on average, and can be calculated as 
\begin{align*}
    \frac1{N(N-1)}\left(\mathrm{sum}\left(\widehat{\mat W_e}\trans{\widehat{\mat W_e}}\right)-N\right)
\end{align*}
where $N\in\N$ is the number of latents.
The $N-1$ and the subtraction of $N$ is in order to ignore the similarity of a latent with itself, since that is always trivially $1$.
Taking the mean naturally loses a lot of nuance, but the metric still has some interesting properties.
A high mean cosine similarity implies that at least most directions are similar to most other directions.
Especially interesting is that a high mean similarity implies a \emph{single} cluster of similar directions, since even if half the directions are similar to each other and the other half are similar to each other, if the two halfs are orthogonal to each other the mean similarity cannot exceed $0.5$.
Unfortunately, this also means that a low mean similarity does not rule out relatively large clusters of similar directions.
Indeed, a mean similarity of $0.1$ would still allow a cluster of equal directions to be $10\%$ of the total number of directions.

\subsubsection{Variance}
The variance is a measure of how spread out the directions are, and is calculated as the trace of the covariance matrix of either $\mat W_e$ or $\widehat{\mat W_e}$.
The former we call the \emph{raw variance} and the latter the \emph{normalized variance}.
The main difference between the two is that the raw variance is sensitive to the magnitude of the directions, while the normalized variance is not.

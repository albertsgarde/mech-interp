\documentclass[../main.tex]{subfiles}


\begin{document}

\section{Preliminaries}
Before we can dive into the specifics of the methods used in this thesis, 
we need to establish some basic concepts and terminology.
Firstly, throughout this section, we assume there is some fixed 
transformer language model \citep{vaswani_attention_2023} 
which we want to interpret.
We will refer to this as the \emph{original language model}.
We will also assume that we have a set of text samples 
which we use to train and test the methods.
We will refer to these as the \emph{dataset}.

Next we introduce the unit of interpretability, namely \emph{features}. 
In the most general form, a feature is any function of the activations 
of a model, but in the context of this thesis we will focus on two 
specific types.
These are individual neurons or activations which can be read directly 
from the model, and neurons in the hidden layer of 
a sparse autoencoder \citep{conmy_towards_2023}.
We will generally refer to these as \emph{neurons} and 
\emph{features} or \emph{features of an SAE} respectively.

Since we interpret features based on their value on various inputs, 
it is useful to have notation for this.
Taking inspiration from \citet{foote_neuron_2023} 
but diverging significantly, we use the function
\begin{align*}
    a\in\mathcal{F}\times\mathcal{S}\times\N\to\R
\end{align*} 
where $\mathcal{F}$ is the set of possible features and
$\mathcal{S}$ is the set of possible token strings.
$a$ then takes a feature, a token string, and an index, and returns 
the value of the feature on the token string at the given index.
For example, if $f$ is the 423rd neuron in the second MLP layer of 
the \verb|gpt2-small| model, $s$ is the token string 
\verb="<|BOS|>","The"," quick"," brown"," fox"=
then $a(f,s,3)$ would be the value of the 423rd neuron on the token 
"brown" when \verb|gpt2-small| processes the string "The quick brown fox".

\section{MAS}
The first method we will use is that of \emph{maximum activating samples}.
It consists of finding the samples which maximally activate a given feature.
The hope is that these samples will give us some insight 
into what the feature is doing.
Mathematically this can be written as
\begin{align*}
    \argmax_{\substack{s\in\mathcal{D}\\ \text{top }k}}\left(\max_{i\leq|s|}\left(a(f,s,i)\right)\right)
\end{align*}
for some feature $f\in\mathcal{F}$ and dataset $\mathcal{D}\subseteq\mathcal{S}$.
This method has faced criticism \citep{bolukbasi_interpretability_2021} 
and should probably not be used in isolation, but it can be a useful
starting point for understanding the behaviour of a feature.

Calculation of the maximum activating samples is rather simple.
Simply iterate over the dataset 
and for each of the $n$ features of interest 
keep track of the $k$ samples which maximally activate the feature.
Here "keep track" means storing the token that causes the activation and 
surrounding $c$ tokens of context along with the activations on that context.
This can be done for the entire model with a single pass over the dataset.
The more of the dataset is used, the "better" the found samples will be, 
but the computational cost will of course increase linearly.

\subbib{../main}
\end{document}